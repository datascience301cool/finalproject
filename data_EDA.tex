% Options for packages loaded elsewhere
\PassOptionsToPackage{unicode}{hyperref}
\PassOptionsToPackage{hyphens}{url}
%
\documentclass[
]{article}
\usepackage{lmodern}
\usepackage{amsmath}
\usepackage{ifxetex,ifluatex}
\ifnum 0\ifxetex 1\fi\ifluatex 1\fi=0 % if pdftex
  \usepackage[T1]{fontenc}
  \usepackage[utf8]{inputenc}
  \usepackage{textcomp} % provide euro and other symbols
  \usepackage{amssymb}
\else % if luatex or xetex
  \usepackage{unicode-math}
  \defaultfontfeatures{Scale=MatchLowercase}
  \defaultfontfeatures[\rmfamily]{Ligatures=TeX,Scale=1}
\fi
% Use upquote if available, for straight quotes in verbatim environments
\IfFileExists{upquote.sty}{\usepackage{upquote}}{}
\IfFileExists{microtype.sty}{% use microtype if available
  \usepackage[]{microtype}
  \UseMicrotypeSet[protrusion]{basicmath} % disable protrusion for tt fonts
}{}
\makeatletter
\@ifundefined{KOMAClassName}{% if non-KOMA class
  \IfFileExists{parskip.sty}{%
    \usepackage{parskip}
  }{% else
    \setlength{\parindent}{0pt}
    \setlength{\parskip}{6pt plus 2pt minus 1pt}}
}{% if KOMA class
  \KOMAoptions{parskip=half}}
\makeatother
\usepackage{xcolor}
\IfFileExists{xurl.sty}{\usepackage{xurl}}{} % add URL line breaks if available
\IfFileExists{bookmark.sty}{\usepackage{bookmark}}{\usepackage{hyperref}}
\hypersetup{
  pdftitle={Data Memo},
  pdfauthor={Edwin Chalas Cuevas, Preston Chan, Lauren Caldrone, Joshua Levitas},
  hidelinks,
  pdfcreator={LaTeX via pandoc}}
\urlstyle{same} % disable monospaced font for URLs
\usepackage[margin=1in]{geometry}
\usepackage{color}
\usepackage{fancyvrb}
\newcommand{\VerbBar}{|}
\newcommand{\VERB}{\Verb[commandchars=\\\{\}]}
\DefineVerbatimEnvironment{Highlighting}{Verbatim}{commandchars=\\\{\}}
% Add ',fontsize=\small' for more characters per line
\usepackage{framed}
\definecolor{shadecolor}{RGB}{248,248,248}
\newenvironment{Shaded}{\begin{snugshade}}{\end{snugshade}}
\newcommand{\AlertTok}[1]{\textcolor[rgb]{0.94,0.16,0.16}{#1}}
\newcommand{\AnnotationTok}[1]{\textcolor[rgb]{0.56,0.35,0.01}{\textbf{\textit{#1}}}}
\newcommand{\AttributeTok}[1]{\textcolor[rgb]{0.77,0.63,0.00}{#1}}
\newcommand{\BaseNTok}[1]{\textcolor[rgb]{0.00,0.00,0.81}{#1}}
\newcommand{\BuiltInTok}[1]{#1}
\newcommand{\CharTok}[1]{\textcolor[rgb]{0.31,0.60,0.02}{#1}}
\newcommand{\CommentTok}[1]{\textcolor[rgb]{0.56,0.35,0.01}{\textit{#1}}}
\newcommand{\CommentVarTok}[1]{\textcolor[rgb]{0.56,0.35,0.01}{\textbf{\textit{#1}}}}
\newcommand{\ConstantTok}[1]{\textcolor[rgb]{0.00,0.00,0.00}{#1}}
\newcommand{\ControlFlowTok}[1]{\textcolor[rgb]{0.13,0.29,0.53}{\textbf{#1}}}
\newcommand{\DataTypeTok}[1]{\textcolor[rgb]{0.13,0.29,0.53}{#1}}
\newcommand{\DecValTok}[1]{\textcolor[rgb]{0.00,0.00,0.81}{#1}}
\newcommand{\DocumentationTok}[1]{\textcolor[rgb]{0.56,0.35,0.01}{\textbf{\textit{#1}}}}
\newcommand{\ErrorTok}[1]{\textcolor[rgb]{0.64,0.00,0.00}{\textbf{#1}}}
\newcommand{\ExtensionTok}[1]{#1}
\newcommand{\FloatTok}[1]{\textcolor[rgb]{0.00,0.00,0.81}{#1}}
\newcommand{\FunctionTok}[1]{\textcolor[rgb]{0.00,0.00,0.00}{#1}}
\newcommand{\ImportTok}[1]{#1}
\newcommand{\InformationTok}[1]{\textcolor[rgb]{0.56,0.35,0.01}{\textbf{\textit{#1}}}}
\newcommand{\KeywordTok}[1]{\textcolor[rgb]{0.13,0.29,0.53}{\textbf{#1}}}
\newcommand{\NormalTok}[1]{#1}
\newcommand{\OperatorTok}[1]{\textcolor[rgb]{0.81,0.36,0.00}{\textbf{#1}}}
\newcommand{\OtherTok}[1]{\textcolor[rgb]{0.56,0.35,0.01}{#1}}
\newcommand{\PreprocessorTok}[1]{\textcolor[rgb]{0.56,0.35,0.01}{\textit{#1}}}
\newcommand{\RegionMarkerTok}[1]{#1}
\newcommand{\SpecialCharTok}[1]{\textcolor[rgb]{0.00,0.00,0.00}{#1}}
\newcommand{\SpecialStringTok}[1]{\textcolor[rgb]{0.31,0.60,0.02}{#1}}
\newcommand{\StringTok}[1]{\textcolor[rgb]{0.31,0.60,0.02}{#1}}
\newcommand{\VariableTok}[1]{\textcolor[rgb]{0.00,0.00,0.00}{#1}}
\newcommand{\VerbatimStringTok}[1]{\textcolor[rgb]{0.31,0.60,0.02}{#1}}
\newcommand{\WarningTok}[1]{\textcolor[rgb]{0.56,0.35,0.01}{\textbf{\textit{#1}}}}
\usepackage{longtable,booktabs}
\usepackage{calc} % for calculating minipage widths
% Correct order of tables after \paragraph or \subparagraph
\usepackage{etoolbox}
\makeatletter
\patchcmd\longtable{\par}{\if@noskipsec\mbox{}\fi\par}{}{}
\makeatother
% Allow footnotes in longtable head/foot
\IfFileExists{footnotehyper.sty}{\usepackage{footnotehyper}}{\usepackage{footnote}}
\makesavenoteenv{longtable}
\usepackage{graphicx}
\makeatletter
\def\maxwidth{\ifdim\Gin@nat@width>\linewidth\linewidth\else\Gin@nat@width\fi}
\def\maxheight{\ifdim\Gin@nat@height>\textheight\textheight\else\Gin@nat@height\fi}
\makeatother
% Scale images if necessary, so that they will not overflow the page
% margins by default, and it is still possible to overwrite the defaults
% using explicit options in \includegraphics[width, height, ...]{}
\setkeys{Gin}{width=\maxwidth,height=\maxheight,keepaspectratio}
% Set default figure placement to htbp
\makeatletter
\def\fps@figure{htbp}
\makeatother
\setlength{\emergencystretch}{3em} % prevent overfull lines
\providecommand{\tightlist}{%
  \setlength{\itemsep}{0pt}\setlength{\parskip}{0pt}}
\setcounter{secnumdepth}{-\maxdimen} % remove section numbering
\ifluatex
  \usepackage{selnolig}  % disable illegal ligatures
\fi

\title{Data Memo}
\author{Edwin Chalas Cuevas, Preston Chan, Lauren Caldrone, Joshua
Levitas}
\date{4/21/2021}

\begin{document}
\maketitle

{
\setcounter{tocdepth}{2}
\tableofcontents
}
\hypertarget{read-in-the-data}{%
\subsection{Read In the Data}\label{read-in-the-data}}

\begin{Shaded}
\begin{Highlighting}[]
\NormalTok{vehicles }\OtherTok{\textless{}{-}} \FunctionTok{read\_csv}\NormalTok{(}\StringTok{"data/vehicles.csv"}\NormalTok{) }\SpecialCharTok{\%\textgreater{}\%} 
  \FunctionTok{clean\_names}\NormalTok{()}
\end{Highlighting}
\end{Shaded}

\begin{verbatim}
## 
## -- Column specification --------------------------------------------------------
## cols(
##   .default = col_character(),
##   id = col_double(),
##   price = col_double(),
##   year = col_double(),
##   odometer = col_double(),
##   county = col_logical(),
##   lat = col_double(),
##   long = col_double()
## )
## i Use `spec()` for the full column specifications.
\end{verbatim}

\hypertarget{eda}{%
\subsection{EDA}\label{eda}}

Before beginning the EDA, I simplified the data set by removing
irrelevant columns These columns include things likethe URL to the
listing, the listing ID, the url to the listing's image, etc. These
columns will have no predictive power.

\begin{Shaded}
\begin{Highlighting}[]
\NormalTok{vehicles }\OtherTok{\textless{}{-}}\NormalTok{ vehicles }\SpecialCharTok{\%\textgreater{}\%} 
  \FunctionTok{select}\NormalTok{(}\SpecialCharTok{{-}}\FunctionTok{c}\NormalTok{(id, url, region\_url, vin, image\_url, description))}
\end{Highlighting}
\end{Shaded}

After doing this, I began the EDA by performing an assurance check of
the data using \texttt{skimr::skim\_without\_charts()}.

\begin{Shaded}
\begin{Highlighting}[]
\NormalTok{skimr}\SpecialCharTok{::}\FunctionTok{skim\_without\_charts}\NormalTok{(vehicles)}
\end{Highlighting}
\end{Shaded}

\begin{longtable}[]{@{}ll@{}}
\caption{Data summary}\tabularnewline
\toprule
\endhead
Name & vehicles\tabularnewline
Number of rows & 441396\tabularnewline
Number of columns & 19\tabularnewline
\_\_\_\_\_\_\_\_\_\_\_\_\_\_\_\_\_\_\_\_\_\_\_ &\tabularnewline
Column type frequency: &\tabularnewline
character & 13\tabularnewline
logical & 1\tabularnewline
numeric & 5\tabularnewline
\_\_\_\_\_\_\_\_\_\_\_\_\_\_\_\_\_\_\_\_\_\_\_\_ &\tabularnewline
Group variables & None\tabularnewline
\bottomrule
\end{longtable}

\textbf{Variable type: character}

\begin{longtable}[]{@{}lrrrrrrr@{}}
\toprule
skim\_variable & n\_missing & complete\_rate & min & max & empty &
n\_unique & whitespace\tabularnewline
\midrule
\endhead
region & 0 & 1.00 & 4 & 26 & 0 & 404 & 0\tabularnewline
manufacturer & 18377 & 0.96 & 3 & 15 & 0 & 42 & 0\tabularnewline
model & 5338 & 0.99 & 1 & 203 & 0 & 30083 & 0\tabularnewline
condition & 183842 & 0.58 & 3 & 9 & 0 & 6 & 0\tabularnewline
cylinders & 188165 & 0.57 & 5 & 12 & 0 & 8 & 0\tabularnewline
fuel & 2881 & 0.99 & 3 & 8 & 0 & 5 & 0\tabularnewline
title\_status & 8945 & 0.98 & 4 & 10 & 0 & 6 & 0\tabularnewline
transmission & 2627 & 0.99 & 5 & 9 & 0 & 3 & 0\tabularnewline
drive & 133649 & 0.70 & 3 & 3 & 0 & 3 & 0\tabularnewline
size & 315584 & 0.29 & 7 & 11 & 0 & 4 & 0\tabularnewline
type & 95349 & 0.78 & 3 & 11 & 0 & 13 & 0\tabularnewline
paint\_color & 133022 & 0.70 & 3 & 6 & 0 & 12 & 0\tabularnewline
state & 0 & 1.00 & 2 & 2 & 0 & 51 & 0\tabularnewline
\bottomrule
\end{longtable}

\textbf{Variable type: logical}

\begin{longtable}[]{@{}lrrrl@{}}
\toprule
skim\_variable & n\_missing & complete\_rate & mean &
count\tabularnewline
\midrule
\endhead
county & 441396 & 0 & NaN & :\tabularnewline
\bottomrule
\end{longtable}

\textbf{Variable type: numeric}

\begin{longtable}[]{@{}lrrrrrrrrr@{}}
\toprule
\begin{minipage}[b]{(\columnwidth - 9\tabcolsep) * \real{0.13}}\raggedright
skim\_variable\strut
\end{minipage} &
\begin{minipage}[b]{(\columnwidth - 9\tabcolsep) * \real{0.09}}\raggedleft
n\_missing\strut
\end{minipage} &
\begin{minipage}[b]{(\columnwidth - 9\tabcolsep) * \real{0.13}}\raggedleft
complete\_rate\strut
\end{minipage} &
\begin{minipage}[b]{(\columnwidth - 9\tabcolsep) * \real{0.09}}\raggedleft
mean\strut
\end{minipage} &
\begin{minipage}[b]{(\columnwidth - 9\tabcolsep) * \real{0.11}}\raggedleft
sd\strut
\end{minipage} &
\begin{minipage}[b]{(\columnwidth - 9\tabcolsep) * \real{0.07}}\raggedleft
p0\strut
\end{minipage} &
\begin{minipage}[b]{(\columnwidth - 9\tabcolsep) * \real{0.08}}\raggedleft
p25\strut
\end{minipage} &
\begin{minipage}[b]{(\columnwidth - 9\tabcolsep) * \real{0.08}}\raggedleft
p50\strut
\end{minipage} &
\begin{minipage}[b]{(\columnwidth - 9\tabcolsep) * \real{0.09}}\raggedleft
p75\strut
\end{minipage} &
\begin{minipage}[b]{(\columnwidth - 9\tabcolsep) * \real{0.12}}\raggedleft
p100\strut
\end{minipage}\tabularnewline
\midrule
\endhead
\begin{minipage}[t]{(\columnwidth - 9\tabcolsep) * \real{0.13}}\raggedright
price\strut
\end{minipage} &
\begin{minipage}[t]{(\columnwidth - 9\tabcolsep) * \real{0.09}}\raggedleft
0\strut
\end{minipage} &
\begin{minipage}[t]{(\columnwidth - 9\tabcolsep) * \real{0.13}}\raggedleft
1.00\strut
\end{minipage} &
\begin{minipage}[t]{(\columnwidth - 9\tabcolsep) * \real{0.09}}\raggedleft
64002.19\strut
\end{minipage} &
\begin{minipage}[t]{(\columnwidth - 9\tabcolsep) * \real{0.11}}\raggedleft
11092408.53\strut
\end{minipage} &
\begin{minipage}[t]{(\columnwidth - 9\tabcolsep) * \real{0.07}}\raggedleft
0.00\strut
\end{minipage} &
\begin{minipage}[t]{(\columnwidth - 9\tabcolsep) * \real{0.08}}\raggedleft
5950.00\strut
\end{minipage} &
\begin{minipage}[t]{(\columnwidth - 9\tabcolsep) * \real{0.08}}\raggedleft
13477.50\strut
\end{minipage} &
\begin{minipage}[t]{(\columnwidth - 9\tabcolsep) * \real{0.09}}\raggedleft
24999.00\strut
\end{minipage} &
\begin{minipage}[t]{(\columnwidth - 9\tabcolsep) * \real{0.12}}\raggedleft
3.736929e+09\strut
\end{minipage}\tabularnewline
\begin{minipage}[t]{(\columnwidth - 9\tabcolsep) * \real{0.13}}\raggedright
year\strut
\end{minipage} &
\begin{minipage}[t]{(\columnwidth - 9\tabcolsep) * \real{0.09}}\raggedleft
1037\strut
\end{minipage} &
\begin{minipage}[t]{(\columnwidth - 9\tabcolsep) * \real{0.13}}\raggedleft
1.00\strut
\end{minipage} &
\begin{minipage}[t]{(\columnwidth - 9\tabcolsep) * \real{0.09}}\raggedleft
2011.49\strut
\end{minipage} &
\begin{minipage}[t]{(\columnwidth - 9\tabcolsep) * \real{0.11}}\raggedleft
9.27\strut
\end{minipage} &
\begin{minipage}[t]{(\columnwidth - 9\tabcolsep) * \real{0.07}}\raggedleft
1900.00\strut
\end{minipage} &
\begin{minipage}[t]{(\columnwidth - 9\tabcolsep) * \real{0.08}}\raggedleft
2009.00\strut
\end{minipage} &
\begin{minipage}[t]{(\columnwidth - 9\tabcolsep) * \real{0.08}}\raggedleft
2014.00\strut
\end{minipage} &
\begin{minipage}[t]{(\columnwidth - 9\tabcolsep) * \real{0.09}}\raggedleft
2017.00\strut
\end{minipage} &
\begin{minipage}[t]{(\columnwidth - 9\tabcolsep) * \real{0.12}}\raggedleft
2.022000e+03\strut
\end{minipage}\tabularnewline
\begin{minipage}[t]{(\columnwidth - 9\tabcolsep) * \real{0.13}}\raggedright
odometer\strut
\end{minipage} &
\begin{minipage}[t]{(\columnwidth - 9\tabcolsep) * \real{0.09}}\raggedleft
4378\strut
\end{minipage} &
\begin{minipage}[t]{(\columnwidth - 9\tabcolsep) * \real{0.13}}\raggedleft
0.99\strut
\end{minipage} &
\begin{minipage}[t]{(\columnwidth - 9\tabcolsep) * \real{0.09}}\raggedleft
225255.58\strut
\end{minipage} &
\begin{minipage}[t]{(\columnwidth - 9\tabcolsep) * \real{0.11}}\raggedleft
16015357.75\strut
\end{minipage} &
\begin{minipage}[t]{(\columnwidth - 9\tabcolsep) * \real{0.07}}\raggedleft
0.00\strut
\end{minipage} &
\begin{minipage}[t]{(\columnwidth - 9\tabcolsep) * \real{0.08}}\raggedleft
39427.00\strut
\end{minipage} &
\begin{minipage}[t]{(\columnwidth - 9\tabcolsep) * \real{0.08}}\raggedleft
85000.00\strut
\end{minipage} &
\begin{minipage}[t]{(\columnwidth - 9\tabcolsep) * \real{0.09}}\raggedleft
131000.00\strut
\end{minipage} &
\begin{minipage}[t]{(\columnwidth - 9\tabcolsep) * \real{0.12}}\raggedleft
2.000799e+09\strut
\end{minipage}\tabularnewline
\begin{minipage}[t]{(\columnwidth - 9\tabcolsep) * \real{0.13}}\raggedright
lat\strut
\end{minipage} &
\begin{minipage}[t]{(\columnwidth - 9\tabcolsep) * \real{0.09}}\raggedleft
6727\strut
\end{minipage} &
\begin{minipage}[t]{(\columnwidth - 9\tabcolsep) * \real{0.13}}\raggedleft
0.98\strut
\end{minipage} &
\begin{minipage}[t]{(\columnwidth - 9\tabcolsep) * \real{0.09}}\raggedleft
38.47\strut
\end{minipage} &
\begin{minipage}[t]{(\columnwidth - 9\tabcolsep) * \real{0.11}}\raggedleft
5.86\strut
\end{minipage} &
\begin{minipage}[t]{(\columnwidth - 9\tabcolsep) * \real{0.07}}\raggedleft
-73.08\strut
\end{minipage} &
\begin{minipage}[t]{(\columnwidth - 9\tabcolsep) * \real{0.08}}\raggedleft
34.43\strut
\end{minipage} &
\begin{minipage}[t]{(\columnwidth - 9\tabcolsep) * \real{0.08}}\raggedleft
39.16\strut
\end{minipage} &
\begin{minipage}[t]{(\columnwidth - 9\tabcolsep) * \real{0.09}}\raggedleft
42.35\strut
\end{minipage} &
\begin{minipage}[t]{(\columnwidth - 9\tabcolsep) * \real{0.12}}\raggedleft
8.462000e+01\strut
\end{minipage}\tabularnewline
\begin{minipage}[t]{(\columnwidth - 9\tabcolsep) * \real{0.13}}\raggedright
long\strut
\end{minipage} &
\begin{minipage}[t]{(\columnwidth - 9\tabcolsep) * \real{0.09}}\raggedleft
6727\strut
\end{minipage} &
\begin{minipage}[t]{(\columnwidth - 9\tabcolsep) * \real{0.13}}\raggedleft
0.98\strut
\end{minipage} &
\begin{minipage}[t]{(\columnwidth - 9\tabcolsep) * \real{0.09}}\raggedleft
-94.57\strut
\end{minipage} &
\begin{minipage}[t]{(\columnwidth - 9\tabcolsep) * \real{0.11}}\raggedleft
18.24\strut
\end{minipage} &
\begin{minipage}[t]{(\columnwidth - 9\tabcolsep) * \real{0.07}}\raggedleft
-175.32\strut
\end{minipage} &
\begin{minipage}[t]{(\columnwidth - 9\tabcolsep) * \real{0.08}}\raggedleft
-111.80\strut
\end{minipage} &
\begin{minipage}[t]{(\columnwidth - 9\tabcolsep) * \real{0.08}}\raggedleft
-88.17\strut
\end{minipage} &
\begin{minipage}[t]{(\columnwidth - 9\tabcolsep) * \real{0.09}}\raggedleft
-80.83\strut
\end{minipage} &
\begin{minipage}[t]{(\columnwidth - 9\tabcolsep) * \real{0.12}}\raggedleft
1.738900e+02\strut
\end{minipage}\tabularnewline
\bottomrule
\end{longtable}

The main takeaway from the results was that a large handful of the
variables had missingness, which was not surprising considering the
massive size of the data set. Some of the variables that seemed to be
important predictors (based on my intuition) had a somewhat high
proportion of missingness, namely \texttt{condition},
\texttt{cylinders}, \texttt{drive}, \texttt{size}, and \texttt{type}.
The proportion of missing values in these columns was 42\%, 37\%, 41\%,
29\%, 70\%, and 25\%, respectively. Most of the other columns had very
low missingness. Another thing I noticed was that \texttt{state} had 51
unique vales and \texttt{region} had 405 unique values. I knew that this
would cause an issue later on when creating a recipe and turning these
variables into dummy variables since it would produce far too many
columns, so I decided to classify the states into 5 regions:
\texttt{northeast}, \texttt{southeast}, \texttt{midwest},
\texttt{southwest}, and \texttt{west}.

\begin{Shaded}
\begin{Highlighting}[]
\NormalTok{vehicles }\OtherTok{\textless{}{-}}\NormalTok{ vehicles }\SpecialCharTok{\%\textgreater{}\%} 
  \FunctionTok{mutate}\NormalTok{(}\AttributeTok{state\_region =} 
           \FunctionTok{case\_when}\NormalTok{(}
\NormalTok{             state }\SpecialCharTok{\%in\%} \FunctionTok{c}\NormalTok{(}\StringTok{"wv"}\NormalTok{, }\StringTok{"va"}\NormalTok{, }\StringTok{"ky"}\NormalTok{, }\StringTok{"nc"}\NormalTok{, }\StringTok{"sc"}\NormalTok{, }\StringTok{"tn"}\NormalTok{, }\StringTok{"ar"}\NormalTok{, }\StringTok{"la"}\NormalTok{, }\StringTok{"al"}\NormalTok{, }\StringTok{"ms"}\NormalTok{, }\StringTok{"ga"}\NormalTok{, }\StringTok{"fl"}\NormalTok{) }\SpecialCharTok{\textasciitilde{}} \StringTok{"southeast"}\NormalTok{, }
\NormalTok{             state }\SpecialCharTok{\%in\%} \FunctionTok{c}\NormalTok{(}\StringTok{"me"}\NormalTok{, }\StringTok{"vt"}\NormalTok{, }\StringTok{"nh"}\NormalTok{, }\StringTok{"ma"}\NormalTok{, }\StringTok{"ct"}\NormalTok{, }\StringTok{"ri"}\NormalTok{, }\StringTok{"ny"}\NormalTok{, }\StringTok{"nj"}\NormalTok{, }\StringTok{"pa"}\NormalTok{, }\StringTok{"de"}\NormalTok{, }\StringTok{"md"}\NormalTok{, }\StringTok{"dc"}\NormalTok{) }\SpecialCharTok{\textasciitilde{}} \StringTok{"northeast"}\NormalTok{, }
\NormalTok{             state }\SpecialCharTok{\%in\%} \FunctionTok{c}\NormalTok{(}\StringTok{"nd"}\NormalTok{, }\StringTok{"sd"}\NormalTok{, }\StringTok{"mn"}\NormalTok{, }\StringTok{"wi"}\NormalTok{, }\StringTok{"ia"}\NormalTok{, }\StringTok{"ne"}\NormalTok{, }\StringTok{"ks"}\NormalTok{, }\StringTok{"mo"}\NormalTok{, }\StringTok{"il"}\NormalTok{, }\StringTok{"in"}\NormalTok{, }\StringTok{"mi"}\NormalTok{, }\StringTok{"oh"}\NormalTok{) }\SpecialCharTok{\textasciitilde{}} \StringTok{"midwest"}\NormalTok{, }
\NormalTok{             state }\SpecialCharTok{\%in\%} \FunctionTok{c}\NormalTok{(}\StringTok{"ok"}\NormalTok{, }\StringTok{"tx"}\NormalTok{, }\StringTok{"nm"}\NormalTok{, }\StringTok{"az"}\NormalTok{) }\SpecialCharTok{\textasciitilde{}} \StringTok{"southwest"}\NormalTok{, }
\NormalTok{             state }\SpecialCharTok{\%in\%} \FunctionTok{c}\NormalTok{(}\StringTok{"ak"}\NormalTok{, }\StringTok{"hi"}\NormalTok{, }\StringTok{"wa"}\NormalTok{, }\StringTok{"or"}\NormalTok{, }\StringTok{"id"}\NormalTok{, }\StringTok{"mt"}\NormalTok{, }\StringTok{"wy"}\NormalTok{, }\StringTok{"co"}\NormalTok{, }\StringTok{"ut"}\NormalTok{, }\StringTok{"nv"}\NormalTok{, }\StringTok{"ca"}\NormalTok{) }\SpecialCharTok{\textasciitilde{}} \StringTok{"west"}
\NormalTok{           )}
\NormalTok{  )}
\end{Highlighting}
\end{Shaded}

I also noticed that many variables that should be classified as factor
variables were classified as character variables. I easily fixed this
with the following code.

\begin{Shaded}
\begin{Highlighting}[]
\NormalTok{vehicles }\OtherTok{\textless{}{-}}\NormalTok{ vehicles }\SpecialCharTok{\%\textgreater{}\%} 
  \FunctionTok{mutate}\NormalTok{(}\AttributeTok{condition =} \FunctionTok{factor}\NormalTok{(condition)) }\SpecialCharTok{\%\textgreater{}\%} 
  \FunctionTok{mutate}\NormalTok{(}\AttributeTok{cylinders =} \FunctionTok{factor}\NormalTok{(cylinders)) }\SpecialCharTok{\%\textgreater{}\%} 
  \FunctionTok{mutate}\NormalTok{(}\AttributeTok{fuel =} \FunctionTok{factor}\NormalTok{(fuel)) }\SpecialCharTok{\%\textgreater{}\%} 
  \FunctionTok{mutate}\NormalTok{(}\AttributeTok{title\_status =} \FunctionTok{factor}\NormalTok{(title\_status)) }\SpecialCharTok{\%\textgreater{}\%} 
  \FunctionTok{mutate}\NormalTok{(}\AttributeTok{transmission =} \FunctionTok{factor}\NormalTok{(transmission)) }\SpecialCharTok{\%\textgreater{}\%} 
  \FunctionTok{mutate}\NormalTok{(}\AttributeTok{drive =} \FunctionTok{factor}\NormalTok{(drive)) }\SpecialCharTok{\%\textgreater{}\%} 
  \FunctionTok{mutate}\NormalTok{(}\AttributeTok{size =} \FunctionTok{factor}\NormalTok{(size)) }\SpecialCharTok{\%\textgreater{}\%} 
  \FunctionTok{mutate}\NormalTok{(}\AttributeTok{type =} \FunctionTok{factor}\NormalTok{(type)) }\SpecialCharTok{\%\textgreater{}\%} 
  \FunctionTok{mutate}\NormalTok{(}\AttributeTok{paint\_color =} \FunctionTok{factor}\NormalTok{(paint\_color)) }\SpecialCharTok{\%\textgreater{}\%} 
  \FunctionTok{mutate}\NormalTok{(}\AttributeTok{state\_region =} \FunctionTok{factor}\NormalTok{(state\_region)) }\SpecialCharTok{\%\textgreater{}\%} 
  \FunctionTok{mutate}\NormalTok{(}\AttributeTok{manufacturer =} \FunctionTok{factor}\NormalTok{(manufacturer))}
\end{Highlighting}
\end{Shaded}

With the basic data cleaning done, I proceeded to conduct an EDA on the
data. I started by focusing on the outcome variable, \texttt{price}. I
first inspected its distribution.

\begin{Shaded}
\begin{Highlighting}[]
\NormalTok{plot1 }\OtherTok{\textless{}{-}}\NormalTok{ vehicles }\SpecialCharTok{\%\textgreater{}\%} 
  \FunctionTok{ggplot}\NormalTok{(}\FunctionTok{aes}\NormalTok{(}\AttributeTok{x =}\NormalTok{ price)) }\SpecialCharTok{+} 
  \FunctionTok{geom\_freqpoly}\NormalTok{() }\SpecialCharTok{+} 
  \FunctionTok{labs}\NormalTok{(}\AttributeTok{title =} \StringTok{"Frequency Polygon"}\NormalTok{)}
\NormalTok{plot2 }\OtherTok{\textless{}{-}}\NormalTok{ vehicles }\SpecialCharTok{\%\textgreater{}\%} 
  \FunctionTok{ggplot}\NormalTok{(}\FunctionTok{aes}\NormalTok{(}\AttributeTok{x =}\NormalTok{ price)) }\SpecialCharTok{+} 
  \FunctionTok{geom\_histogram}\NormalTok{() }\SpecialCharTok{+} 
  \FunctionTok{labs}\NormalTok{(}\AttributeTok{title =} \StringTok{"Histogram"}\NormalTok{)}
\NormalTok{plot3 }\OtherTok{\textless{}{-}}\NormalTok{ vehicles }\SpecialCharTok{\%\textgreater{}\%} 
  \FunctionTok{ggplot}\NormalTok{(}\FunctionTok{aes}\NormalTok{(}\AttributeTok{x =}\NormalTok{ price)) }\SpecialCharTok{+} 
  \FunctionTok{geom\_density}\NormalTok{() }\SpecialCharTok{+} 
  \FunctionTok{labs}\NormalTok{(}\AttributeTok{title =} \StringTok{"Density Plot"}\NormalTok{)}
\NormalTok{plot4 }\OtherTok{\textless{}{-}}\NormalTok{ vehicles }\SpecialCharTok{\%\textgreater{}\%} 
  \FunctionTok{ggplot}\NormalTok{(}\FunctionTok{aes}\NormalTok{(}\AttributeTok{x =}\NormalTok{ price)) }\SpecialCharTok{+} 
  \FunctionTok{geom\_boxplot}\NormalTok{() }\SpecialCharTok{+} 
  \FunctionTok{labs}\NormalTok{(}\AttributeTok{title =} \StringTok{"Box Plot"}\NormalTok{)}
\NormalTok{(plot1 }\SpecialCharTok{+}\NormalTok{ plot2) }\SpecialCharTok{/}\NormalTok{ (plot3 }\SpecialCharTok{+}\NormalTok{ plot4)}
\end{Highlighting}
\end{Shaded}

\includegraphics{data_EDA_files/figure-latex/unnamed-chunk-6-1.pdf}

Based on these plots, it was difficult to see the distribution because
the distribution was so skewed to the right. I decided to see if I could
filter out any potential outliers to make it easier to see the
distribution. I first looked at how many vehicles were priced at over
\$100,000, since this seemed like a price would be considered an outlier
among used cars.

\begin{Shaded}
\begin{Highlighting}[]
\NormalTok{vehicles }\SpecialCharTok{\%\textgreater{}\%} 
  \FunctionTok{filter}\NormalTok{(price }\SpecialCharTok{\textgreater{}} \DecValTok{100000}\NormalTok{) }\SpecialCharTok{\%\textgreater{}\%} 
  \FunctionTok{count}\NormalTok{()}
\end{Highlighting}
\end{Shaded}

\begin{verbatim}
## # A tibble: 1 x 1
##       n
##   <int>
## 1   691
\end{verbatim}

Only 596 vehicles out of over 458,000 listings were priced at above
\$100,000. This implied that cars listed for more than \$100,000 were
clearly outliers.

I also decided to look at how many listings were priced at below
\$1,000. This is because earlier, I had gone on Craiglist to look at
cars priced below \$1,000 and discovered that many of these listings
were for cars that could be paid off in monthly increments of a few
hundred dollars. Since these monthly prices do not reflect the full
value of the car, I decided it would be a good idea to filter these out.
I decided to see how many listings would be taken out by this filter.

\begin{Shaded}
\begin{Highlighting}[]
\NormalTok{vehicles }\SpecialCharTok{\%\textgreater{}\%} 
  \FunctionTok{filter}\NormalTok{(price }\SpecialCharTok{\textless{}} \DecValTok{1000}\NormalTok{) }\SpecialCharTok{\%\textgreater{}\%} 
  \FunctionTok{count}\NormalTok{()}
\end{Highlighting}
\end{Shaded}

\begin{verbatim}
## # A tibble: 1 x 1
##       n
##   <int>
## 1 48206
\end{verbatim}

More than 48,000 listings fell into this price range. This seemed like a
high number, so I decided to look further into this subset of the data.
I looked at the number of cars priced at \$0.

\begin{Shaded}
\begin{Highlighting}[]
\NormalTok{vehicles }\SpecialCharTok{\%\textgreater{}\%} 
  \FunctionTok{filter}\NormalTok{(price }\SpecialCharTok{==} \DecValTok{0}\NormalTok{) }\SpecialCharTok{\%\textgreater{}\%} 
  \FunctionTok{count}\NormalTok{()}
\end{Highlighting}
\end{Shaded}

\begin{verbatim}
## # A tibble: 1 x 1
##       n
##   <int>
## 1 34089
\end{verbatim}

More than 33,000 listings were priced at \$0. Upon looking at such
listings on Craigslist, I discovered that almost all of these listings
were from people who were hiding the real price of the car. In other
words, they set the price to \$0, and then inside the description, they
would include a link to a different website which had the actual price
of the car.

Based on these analyses, I decided that listings priced below \$1,000 or
above \$100,000 were outliers. Thus, I looked at the distribution of
\texttt{price} again after filtering out these outliers.

\begin{Shaded}
\begin{Highlighting}[]
\NormalTok{plot1 }\OtherTok{\textless{}{-}}\NormalTok{ vehicles }\SpecialCharTok{\%\textgreater{}\%} 
  \FunctionTok{filter}\NormalTok{(price }\SpecialCharTok{\textgreater{}} \DecValTok{1000} \SpecialCharTok{\&}\NormalTok{ price }\SpecialCharTok{\textless{}} \DecValTok{100000}\NormalTok{) }\SpecialCharTok{\%\textgreater{}\%} 
  \FunctionTok{ggplot}\NormalTok{(}\FunctionTok{aes}\NormalTok{(}\AttributeTok{x =}\NormalTok{ price)) }\SpecialCharTok{+} 
  \FunctionTok{geom\_freqpoly}\NormalTok{() }\SpecialCharTok{+} 
  \FunctionTok{labs}\NormalTok{(}\AttributeTok{title =} \StringTok{"Frequency Polygon"}\NormalTok{)}
\NormalTok{plot2 }\OtherTok{\textless{}{-}}\NormalTok{ vehicles }\SpecialCharTok{\%\textgreater{}\%} 
  \FunctionTok{filter}\NormalTok{(price }\SpecialCharTok{\textgreater{}} \DecValTok{1000} \SpecialCharTok{\&}\NormalTok{ price }\SpecialCharTok{\textless{}} \DecValTok{100000}\NormalTok{) }\SpecialCharTok{\%\textgreater{}\%} 
  \FunctionTok{ggplot}\NormalTok{(}\FunctionTok{aes}\NormalTok{(}\AttributeTok{x =}\NormalTok{ price)) }\SpecialCharTok{+} 
  \FunctionTok{geom\_histogram}\NormalTok{() }\SpecialCharTok{+} 
  \FunctionTok{labs}\NormalTok{(}\AttributeTok{title =} \StringTok{"Histogram"}\NormalTok{)}
\NormalTok{plot3 }\OtherTok{\textless{}{-}}\NormalTok{ vehicles }\SpecialCharTok{\%\textgreater{}\%} 
  \FunctionTok{filter}\NormalTok{(price }\SpecialCharTok{\textgreater{}} \DecValTok{1000} \SpecialCharTok{\&}\NormalTok{ price }\SpecialCharTok{\textless{}} \DecValTok{100000}\NormalTok{) }\SpecialCharTok{\%\textgreater{}\%} 
  \FunctionTok{ggplot}\NormalTok{(}\FunctionTok{aes}\NormalTok{(}\AttributeTok{x =}\NormalTok{ price)) }\SpecialCharTok{+} 
  \FunctionTok{geom\_density}\NormalTok{() }\SpecialCharTok{+} 
  \FunctionTok{labs}\NormalTok{(}\AttributeTok{title =} \StringTok{"Density Plot"}\NormalTok{)}
\NormalTok{plot4 }\OtherTok{\textless{}{-}}\NormalTok{ vehicles }\SpecialCharTok{\%\textgreater{}\%} 
  \FunctionTok{filter}\NormalTok{(price }\SpecialCharTok{\textgreater{}} \DecValTok{1000} \SpecialCharTok{\&}\NormalTok{ price }\SpecialCharTok{\textless{}} \DecValTok{100000}\NormalTok{) }\SpecialCharTok{\%\textgreater{}\%} 
  \FunctionTok{ggplot}\NormalTok{(}\FunctionTok{aes}\NormalTok{(}\AttributeTok{x =}\NormalTok{ price)) }\SpecialCharTok{+} 
  \FunctionTok{geom\_boxplot}\NormalTok{() }\SpecialCharTok{+} 
  \FunctionTok{labs}\NormalTok{(}\AttributeTok{title =} \StringTok{"Box Plot"}\NormalTok{)}
\NormalTok{(plot1 }\SpecialCharTok{+}\NormalTok{ plot2) }\SpecialCharTok{/}\NormalTok{ (plot3 }\SpecialCharTok{+}\NormalTok{ plot4)}
\end{Highlighting}
\end{Shaded}

\includegraphics{data_EDA_files/figure-latex/unnamed-chunk-10-1.pdf}

The distribution was easier to see than before, but it was still skewed
to the right. Because of this, I decided to log transform \texttt{price}
and look at the distribution again.

\begin{Shaded}
\begin{Highlighting}[]
\NormalTok{plot1 }\OtherTok{\textless{}{-}}\NormalTok{ vehicles }\SpecialCharTok{\%\textgreater{}\%} 
  \FunctionTok{filter}\NormalTok{(price }\SpecialCharTok{\textgreater{}} \DecValTok{1000} \SpecialCharTok{\&}\NormalTok{ price }\SpecialCharTok{\textless{}} \DecValTok{100000}\NormalTok{) }\SpecialCharTok{\%\textgreater{}\%} 
  \FunctionTok{mutate}\NormalTok{(}\AttributeTok{price =} \FunctionTok{log10}\NormalTok{(price)) }\SpecialCharTok{\%\textgreater{}\%} 
  \FunctionTok{ggplot}\NormalTok{(}\FunctionTok{aes}\NormalTok{(}\AttributeTok{x =}\NormalTok{ price)) }\SpecialCharTok{+} 
  \FunctionTok{geom\_freqpoly}\NormalTok{() }\SpecialCharTok{+} 
  \FunctionTok{labs}\NormalTok{(}\AttributeTok{title =} \StringTok{"Frequency Polygon"}\NormalTok{)}
\NormalTok{plot2 }\OtherTok{\textless{}{-}}\NormalTok{ vehicles }\SpecialCharTok{\%\textgreater{}\%} 
  \FunctionTok{filter}\NormalTok{(price }\SpecialCharTok{\textgreater{}} \DecValTok{1000} \SpecialCharTok{\&}\NormalTok{ price }\SpecialCharTok{\textless{}} \DecValTok{100000}\NormalTok{) }\SpecialCharTok{\%\textgreater{}\%} 
  \FunctionTok{mutate}\NormalTok{(}\AttributeTok{price =} \FunctionTok{log10}\NormalTok{(price)) }\SpecialCharTok{\%\textgreater{}\%} 
  \FunctionTok{ggplot}\NormalTok{(}\FunctionTok{aes}\NormalTok{(}\AttributeTok{x =}\NormalTok{ price)) }\SpecialCharTok{+} 
  \FunctionTok{geom\_histogram}\NormalTok{() }\SpecialCharTok{+} 
  \FunctionTok{labs}\NormalTok{(}\AttributeTok{title =} \StringTok{"Histogram"}\NormalTok{)}
\NormalTok{plot3 }\OtherTok{\textless{}{-}}\NormalTok{ vehicles }\SpecialCharTok{\%\textgreater{}\%} 
  \FunctionTok{filter}\NormalTok{(price }\SpecialCharTok{\textgreater{}} \DecValTok{1000} \SpecialCharTok{\&}\NormalTok{ price }\SpecialCharTok{\textless{}} \DecValTok{100000}\NormalTok{) }\SpecialCharTok{\%\textgreater{}\%} 
  \FunctionTok{mutate}\NormalTok{(}\AttributeTok{price =} \FunctionTok{log10}\NormalTok{(price)) }\SpecialCharTok{\%\textgreater{}\%} 
  \FunctionTok{ggplot}\NormalTok{(}\FunctionTok{aes}\NormalTok{(}\AttributeTok{x =}\NormalTok{ price)) }\SpecialCharTok{+} 
  \FunctionTok{geom\_density}\NormalTok{() }\SpecialCharTok{+} 
  \FunctionTok{labs}\NormalTok{(}\AttributeTok{title =} \StringTok{"Density Plot"}\NormalTok{)}
\NormalTok{plot4 }\OtherTok{\textless{}{-}}\NormalTok{ vehicles }\SpecialCharTok{\%\textgreater{}\%} 
  \FunctionTok{filter}\NormalTok{(price }\SpecialCharTok{\textgreater{}} \DecValTok{1000} \SpecialCharTok{\&}\NormalTok{ price }\SpecialCharTok{\textless{}} \DecValTok{100000}\NormalTok{) }\SpecialCharTok{\%\textgreater{}\%} 
  \FunctionTok{mutate}\NormalTok{(}\AttributeTok{price =} \FunctionTok{log10}\NormalTok{(price)) }\SpecialCharTok{\%\textgreater{}\%} 
  \FunctionTok{ggplot}\NormalTok{(}\FunctionTok{aes}\NormalTok{(}\AttributeTok{x =}\NormalTok{ price)) }\SpecialCharTok{+} 
  \FunctionTok{geom\_boxplot}\NormalTok{() }\SpecialCharTok{+} 
  \FunctionTok{labs}\NormalTok{(}\AttributeTok{title =} \StringTok{"Box Plot"}\NormalTok{)}
\NormalTok{(plot1 }\SpecialCharTok{+}\NormalTok{ plot2) }\SpecialCharTok{/}\NormalTok{ (plot3 }\SpecialCharTok{+}\NormalTok{ plot4)}
\end{Highlighting}
\end{Shaded}

\includegraphics{data_EDA_files/figure-latex/unnamed-chunk-11-1.pdf}

This distribution looked much more normal, and the boxplot showed no
outliers. Therefore, I decided to filter out the outliers from the
\texttt{vehicles} data set and log transform \texttt{price}.

\begin{Shaded}
\begin{Highlighting}[]
\NormalTok{vehicles }\OtherTok{\textless{}{-}}\NormalTok{ vehicles }\SpecialCharTok{\%\textgreater{}\%} 
  \FunctionTok{filter}\NormalTok{(price }\SpecialCharTok{\textgreater{}} \DecValTok{1000} \SpecialCharTok{\&}\NormalTok{ price }\SpecialCharTok{\textless{}} \DecValTok{100000}\NormalTok{) }\SpecialCharTok{\%\textgreater{}\%} 
  \FunctionTok{mutate}\NormalTok{(}\AttributeTok{price =} \FunctionTok{log10}\NormalTok{(price))}
\end{Highlighting}
\end{Shaded}

The next thing I wanted to inspect was the correlation of the numeric
x-variables (\texttt{year}, \texttt{odometer}, \texttt{lat}, and
\texttt{long}) with \texttt{price} to see if any of them would be useful
predictors that could be included in the model recipes. I used
\texttt{stats::cor()} to create a correlation table for these variables,
making sure to filter out listings with \texttt{price} greater than
\$1000 and and \texttt{price} less than \$100,000.

\begin{Shaded}
\begin{Highlighting}[]
\NormalTok{vehicles }\SpecialCharTok{\%\textgreater{}\%} 
  \FunctionTok{mutate}\NormalTok{(}\AttributeTok{price =}\NormalTok{ (}\DecValTok{10} \SpecialCharTok{\^{}}\NormalTok{ price)) }\SpecialCharTok{\%\textgreater{}\%} 
  \FunctionTok{select}\NormalTok{(price, year, odometer, lat, long) }\SpecialCharTok{\%\textgreater{}\%} 
  \FunctionTok{cor}\NormalTok{(}\AttributeTok{use =} \StringTok{"complete.obs"}\NormalTok{)}
\end{Highlighting}
\end{Shaded}

\begin{verbatim}
##                 price         year     odometer          lat        long
## price     1.000000000  0.355917437  0.002151397 -0.009935894 -0.09131840
## year      0.355917437  1.000000000  0.003344989 -0.004537745  0.00414594
## odometer  0.002151397  0.003344989  1.000000000 -0.003164886  0.00376453
## lat      -0.009935894 -0.004537745 -0.003164886  1.000000000 -0.13191087
## long     -0.091318397  0.004145940  0.003764530 -0.131910873  1.00000000
\end{verbatim}

From these results it was apparent that the only variable that had a
correlation with \texttt{price} (albeit a small one) was year. The other
variables essentially had no correlation. However, I wanted to see if
there were unusual values in these variables that were decreasing their
correlation with \texttt{price}. In particular, I wanted to look for
unusual values in \texttt{year} and \texttt{odometer} as there cannot be
unusual values for \texttt{lat} (latitude) and \texttt{long} (longitude)
since they are simply geographical coordinates.

I decided to first look at the distribution of \texttt{year} to see if
there were any unusual values affecting its correlation with
\texttt{price}.

\begin{Shaded}
\begin{Highlighting}[]
\NormalTok{vehicles }\SpecialCharTok{\%\textgreater{}\%} 
  \FunctionTok{mutate}\NormalTok{(}\AttributeTok{price =}\NormalTok{ (}\DecValTok{10} \SpecialCharTok{\^{}}\NormalTok{ price)) }\SpecialCharTok{\%\textgreater{}\%} 
  \FunctionTok{ggplot}\NormalTok{(}\FunctionTok{aes}\NormalTok{(year)) }\SpecialCharTok{+} 
  \FunctionTok{geom\_boxplot}\NormalTok{() }\SpecialCharTok{+} 
  \FunctionTok{labs}\NormalTok{(}\AttributeTok{title =} \StringTok{"Distribution of Year"}\NormalTok{)}
\end{Highlighting}
\end{Shaded}

\begin{verbatim}
## Warning: Removed 1012 rows containing non-finite values (stat_boxplot).
\end{verbatim}

\includegraphics{data_EDA_files/figure-latex/unnamed-chunk-14-1.pdf}

From the boxplot, it was clear that there were many outliers, and these
outliers were from years prior to roughly 1993.

I then looked at the distribution of \texttt{odometer} to see if there
were any unusual values affective its correlation with \texttt{price}.

\begin{Shaded}
\begin{Highlighting}[]
\NormalTok{vehicles }\SpecialCharTok{\%\textgreater{}\%} 
  \FunctionTok{mutate}\NormalTok{(}\AttributeTok{price =}\NormalTok{ (}\DecValTok{10} \SpecialCharTok{\^{}}\NormalTok{ price)) }\SpecialCharTok{\%\textgreater{}\%} 
  \FunctionTok{ggplot}\NormalTok{(}\FunctionTok{aes}\NormalTok{(odometer)) }\SpecialCharTok{+} 
  \FunctionTok{geom\_boxplot}\NormalTok{() }\SpecialCharTok{+} 
  \FunctionTok{labs}\NormalTok{(}\AttributeTok{title =} \StringTok{"Distribution of Odometer"}\NormalTok{)}
\end{Highlighting}
\end{Shaded}

\begin{verbatim}
## Warning: Removed 2278 rows containing non-finite values (stat_boxplot).
\end{verbatim}

\includegraphics{data_EDA_files/figure-latex/unnamed-chunk-15-1.pdf}

From the boxplot, it was clear that there were a few extremely large
outliers, which made it difficult to see if there were less extreme
outliers. However, I knew that most cars start to fall apart when they
reach 250,000-300,000 miles, so I decided to see how many cars in the
data set had over 300,000 miles on their \texttt{odometer}.

\begin{Shaded}
\begin{Highlighting}[]
\NormalTok{vehicles }\SpecialCharTok{\%\textgreater{}\%} 
  \FunctionTok{mutate}\NormalTok{(}\AttributeTok{price =}\NormalTok{ (}\DecValTok{10} \SpecialCharTok{\^{}}\NormalTok{ price)) }\SpecialCharTok{\%\textgreater{}\%} 
  \FunctionTok{filter}\NormalTok{(odometer }\SpecialCharTok{\textgreater{}} \DecValTok{300000}\NormalTok{) }\SpecialCharTok{\%\textgreater{}\%} 
  \FunctionTok{count}\NormalTok{()}
\end{Highlighting}
\end{Shaded}

\begin{verbatim}
## # A tibble: 1 x 1
##       n
##   <int>
## 1  2428
\end{verbatim}

Only 2,154 listings out of the 400,000+ listings had an odometer with
over 300,000 miles. Therefore, I considered these as outliers.

Having found the unusual values in \texttt{year} and \texttt{odometer}
that may have been decreasing their correlation with \texttt{price}, I
created a new correlation table in which I filtered out the unusual
values.

\begin{Shaded}
\begin{Highlighting}[]
\NormalTok{vehicles }\SpecialCharTok{\%\textgreater{}\%} 
  \FunctionTok{mutate}\NormalTok{(}\AttributeTok{price =}\NormalTok{ (}\DecValTok{10} \SpecialCharTok{\^{}}\NormalTok{ price)) }\SpecialCharTok{\%\textgreater{}\%} 
  \FunctionTok{filter}\NormalTok{(odometer }\SpecialCharTok{\textless{}} \DecValTok{300000}\NormalTok{) }\SpecialCharTok{\%\textgreater{}\%} 
  \FunctionTok{filter}\NormalTok{(year }\SpecialCharTok{\textgreater{}} \DecValTok{1993}\NormalTok{) }\SpecialCharTok{\%\textgreater{}\%} 
  \FunctionTok{select}\NormalTok{(price, year, odometer, lat, long) }\SpecialCharTok{\%\textgreater{}\%} 
  \FunctionTok{cor}\NormalTok{(}\AttributeTok{use =} \StringTok{"complete.obs"}\NormalTok{)}
\end{Highlighting}
\end{Shaded}

\begin{verbatim}
##                 price        year    odometer          lat        long
## price     1.000000000  0.59629591 -0.54804074 -0.009168215 -0.09662794
## year      0.596295910  1.00000000 -0.65748498 -0.025896601 -0.01945802
## odometer -0.548040745 -0.65748498  1.00000000  0.030680238  0.03442845
## lat      -0.009168215 -0.02589660  0.03068024  1.000000000 -0.13400554
## long     -0.096627944 -0.01945802  0.03442845 -0.134005543  1.00000000
\end{verbatim}

Based on these results, it was clear that \texttt{year} and
\texttt{odometer} actualy have moderate correlations with
\texttt{price}. In other words, they seemed like useful predictors that
could be included in the model recipes.

Furthermore, the EDA revealed that \texttt{year} and \texttt{odometer}
had notable outliers that should be removed from the \texttt{vehicles}
data set. Thus, I used the following code to remove these outliers from
\texttt{vehicles}.

\begin{Shaded}
\begin{Highlighting}[]
\NormalTok{vehicles }\OtherTok{\textless{}{-}}\NormalTok{ vehicles }\SpecialCharTok{\%\textgreater{}\%} 
  \FunctionTok{filter}\NormalTok{(odometer }\SpecialCharTok{\textless{}} \DecValTok{300000}\NormalTok{) }\SpecialCharTok{\%\textgreater{}\%} 
  \FunctionTok{filter}\NormalTok{(year }\SpecialCharTok{\textgreater{}} \DecValTok{1993}\NormalTok{)}
\end{Highlighting}
\end{Shaded}


\end{document}
